\section{A data exploration case study}

% Suppose you want to 
% Different kinds of statistical approach are appropriate to the analysis 
% of different combinations of variables. For example, to test whether a 
% continuous dependent variable (e.g. life-span) varies significantly 
% across a categorical independent variable (e.g. handedness), then a t 
% test might be appropriate. However, a t test would be inappropriate to 
% test whether a continuous dependent variable (e.g. growth rate) varies 
% significantly across a continuous independent variable (e.g. nutrient 
% concentration): here some sort of regression would be required. It is 
% critical that you know what sorts of test are appropriate to different 
% kinds of data.
 % If your dependent variable is count data, random variation in the 
 % variable is likely to be distributed according to the Poisson 
 % distribution, and particular care must be taken in its analysis.
% 
% 


We will use the famous {\it Iris} data set which is available in R (one 
of many) as a case study for data exploration and descriptive 
statistics. This dataset gives the measurements in centimeters of the 
variables sepal length and width and petal length and width, 
respectively, for 50 flowers from each of 3 species of iris: {\it Iris 
setosa}, {\it I. versicolor}, and {\it I. virginica}. All data came 
from the same  pasture, and picked on the same day and measured at the 
same time by the same person with the same apparatus. 

\section{Subsetting Data}
Often, you will need to get only part of a matrix, dataframe, etc. Here 
are ways to subset your data:
\begin{snugshade}
\footnotesize
\begin{verbatim}
> M <- matrix(1:9, 3, 3)
> M
	 [,1] [,2] [,3]
[1,]    1    4    7
[2,]    2    5    8
[3,]    3    6    9
> M[1,1]
[1] 1
> M[1, ]
[1] 1 4 7
> M[1:2, ]
	 [,1] [,2] [,3]
[1,]    1    4    7
[2,]    2    5    8
> M[c(1, 3), 2]
[1] 4 6
> M[-3,]
	 [,1] [,2] [,3]
[1,]    1    4    7
[2,]    2    5    8
> M[-3,-2]
	 [,1] [,2]
[1,]    1    7
[2,]    2    8
> attach(iris)
> iris
	 Sepal.Length Sepal.Width Petal.Length Petal.Width Species
1            5.1         3.5          1.4         0.2 setosa
2            4.9         3.0          1.4         0.2 setosa
3            4.7         3.2          1.3         0.2 setosa
4            4.6         3.1          1.5         0.2 setosa
5            5.0         3.6          1.4         0.2 setosa
6            5.4         3.9          1.7         0.4 setosa
7            4.6         3.4          1.4         0.3 setosa
8            5.0         3.4          1.5         0.2 setosa
9            4.4         2.9          1.4         0.2 setosa
10           4.9         3.1          1.5         0.1 setosa
> iris$Sepal.Width[1:5]
[1] 3.5 3.0 3.2 3.1 3.6
> iris[1:3, c("Sepal.Width","Petal.Width")]
	Sepal.Width Petal.Width
1         3.5         0.2
2         3.0         0.2
3         3.2         0.2
> iris[(Sepal.Width < 3.5) & (Sepal.Length > 7.2),]
 Sepal.Length Sepal.Width Petal.Length Petal.Width Species
106      7.6         3.0          6.6         2.1 virginica
108      7.3         2.9          6.3         1.8 virginica
119      7.7         2.6          6.9         2.3 virginica
123      7.7         2.8          6.7         2.0 virginica
131      7.4         2.8          6.1         1.9 virginica
136      7.7         3.0          6.1         2.3 virginica
\end{verbatim}
\end{snugshade}

\noindent  
\section{Basic Statistics in R}
R contains several functions to perform basic (and advanced!) 
statistics. 
\begin{snugshade} \footnotesize
\begin{verbatim}
> attach(iris)
> summary(iris)
  Sepal.Length    Sepal.Width     Petal.Length  Petal.Width   
 Min.   :4.300   Min.   :2.000   Min.   :1.000 Min.   :0.100  
 1st Qu.:5.100   1st Qu.:2.800   1st Qu.:1.600 1st Qu.:0.300  
 Median :5.800   Median :3.000   Median :4.350 Median :1.300  
 Mean   :5.843   Mean   :3.057   Mean   :3.758 Mean   :1.199  
 3rd Qu.:6.400   3rd Qu.:3.300   3rd Qu.:5.100 3rd Qu.:1.800  
 Max.   :7.900   Max.   :4.400   Max.   :6.900 Max.   :2.500  
       Species  
 setosa    :50  
 versicolor:50  
 virginica :50  
> table(Species)
Species
    setosa versicolor  virginica 
        50         50         50

> table(Species, Petal.Width)
            Petal.Width
Species      0.1 0.2 0.3 0.4 0.5 0.6  1 1.1 1.2 1.3 1.4 ...
  setosa       5  29   7   7   1   1  0   0   0   0   0 ...
  versicolor   0   0   0   0   0   0  7   3   5  13   7 ...
  virginica    0   0   0   0   0   0  0   0   0   0   1 ...
            Petal.Width
Species      2.1 2.2 2.3 2.4 2.5
  setosa       0   0   0   0   0
  versicolor   0   0   0   0   0
  virginica    6   3   8   3   3
> t.test(Sepal.Width[which(Species == "setosa")], 
         Sepal.Width[which(Species == "versicolor")])

	Welch Two Sample t-test

data:  Sepal.Width[which(Species == "setosa")] and 
       Sepal.Width[which(Species == "versicolor")] 
t = 9.455, df = 94.698, p-value = 2.484e-15
alternative hypothesis: true difference in means is
not equal to 0 
95 percent confidence interval:
 0.5198348 0.7961652 
sample estimates:
mean of x mean of y 
    3.428     2.770 
> summary(lm(Sepal.Width ~ Sepal.Length))

Call:
lm(formula = Sepal.Width ~ Sepal.Length)

Residuals:
    Min      1Q  Median      3Q     Max 
-1.1095 -0.2454 -0.0167  0.2763  1.3338 

Coefficients:
             Estimate Std. Error t value Pr(>|t|)    
(Intercept)   3.41895    0.25356   13.48   <2e-16 ***
Sepal.Length -0.06188    0.04297   -1.44    0.152    
---
Signif. codes:  0 ‘***’ 0.001 ‘**’ 0.01 ‘*’ 0.05 ‘.’ 0.1 ‘ ’ 1 

Residual standard error: 0.4343 on 148 degrees of freedom
Multiple R-squared: 0.01382,	Adjusted R-squared: 0.007159 
F-statistic: 2.074 on 1 and 148 DF,  p-value: 0.1519 
\end{verbatim}
\end{snugshade}
