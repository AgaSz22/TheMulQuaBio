\chapter{Introduction}

\begin{quote}{Donald Knuth, 1995: }
{\it Science is what we understand well enough to explain to a 
computer. Art is everything else we do.}
\end{quote}

\section{What is this document about?} 

\noindent This document contains the content of the Imperial College 
London Department of Life Sciences BSc Year 1 (Y1) and 2 (Y2) 
statistics modules, including workshops and practicals. Different 
subsets of the content of this document will be covered in Y1 and Y2. 
You will be given instructions about which sections will be covered in 
your respective (Y1 or Y2) modules. You will learn statistics with R in 
a hands-on on way in this module. A few ``pure'' lectures on statistical 
concepts will also be delivered. 

This document is accompanied by data and code on which you can practice 
your skills in your own time and during the Practical sessions. These 
materials are available (and will be updated regularly) at bitbucket 
(it's like GitHub): \url{https://bitbucket.org/mhasoba/iccompbiostat}.  

{\bf Note that you can download the code, data and everything 
from the bitbucket repository at one go, by going to 
\url{https://bitbucket.org/mhasoba/iccompbiostat}
and then clicking on the ``Download repository'' link. You can then unzip the 
downloaded .zip and grab the files you need.}

These notes and the scripts that you create form a valuable reference 
for using and interpreting statistics. You will be expected to be able 
to use your experience and these notes to complete analyses for 
statistical components of other courses, in the second and third year 
in particular.

\noindent{\color{red} The topics covered here assume that you have 
already worked through at least the basic R sections of the 
{\tt SilBioComp.pdf} document (also available in the same bitbucket 
repository)}

Finally, it is important that you work through the problems in each 
chapter, particularly as some of the questions ask you to find out 
about commands and functions not introduced in the chapter's text 
itself, but which will be relied on in later chapters. By the time you 
have attended the lectures and workshops, and completed the exercises 
in this document, you should able to: 
\begin{compactitem}\itemsep4pt
    \item Use {R} to perform common statistical tests, particularly t, 
    F, and $\chi^{2}$ tests
    \item Develop ability to build, criticize and interpret linear 
    models, especially linear regression and ANOVA. 
    \item Interpret R output (particularly p values) correctly, and use 
    them appropriately in practical write-ups and presentations
\end{compactitem} 

\section{Conventions used in this document}

\noindent You will find all R commandline/console arguments, code  
snippets and output in colored boxes like this:
\begin{lstlisting}
> ls()
\end{lstlisting}
Here $>$ is the R prompt, and will type (or copy-paste, but not 
recommended!) the commands/code that you see from this document into 
the R command line. But please exclude the $>$ as this is just the R 
command prompt! I have included the $>$ prompts in the code shown in 
this document so that you are forced to see what each line is doing. 
Indeed, avoid copying-and-pasting chunks of R code you do not 
understand: blindly shovelling data into a black box and assuming the 
output is correct and meaningful will eventually lead to frustrations, 
and if you are unlucky, embarrassments or even catastrophes!

The content of this document is computer platform (Mac, PC or Linux) 
independent because many of you will probably also later be working 
with R on your personal laptops or desktops. Indeed, 
platform-independence of your statistical analysis is one of the main 
reasons why you are using R! 

Finally, note that:
\begin{compactitem}[$\quad\star$]\itemsep4pt
	\item In all subsequent chapters, lines marked with a star like this 
	are things for you to do.
\end{compactitem}

\section{Readings}

Look up the Readings directory on the bitbucket repository (link 
given above).

\begin{itemize}\itemsep2pt{}
	\item Bolker, B. M.: Ecological Models and Data in R (eBook and 
	Hardcover available).
	\item Beckerman, A. P. \& Petchey, O. L. (2012) Getting started 
	with R: an introduction for biologists. Oxford, Oxford University 
	Press. \\ Good, short, general introduction
	\item Crawley, R. (2013) The R book. 2nd edition. Chichester, Wiley. \\
	Excellent but enormous reference book, code and data available from
	\url{www.bio.ic.ac.uk/research/mjcraw/therbook/index.htm}
	\item Use the internet! Type "R tutorial", and scores will pop up. 
	Choose one that seems the most intuitive to you. 
\end{itemize}
